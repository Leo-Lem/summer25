\documentclass{article}

\usepackage[]{xrcise}

\subject{Empirical Software Engineering}
\semester{Summer 2025}
\author{Leopold Lemmermann}

\begin{document}\createtitle

\sheet[2024]{First Exam}

\begin{exercise}{Research}
  \begin{enumerate}
    \item Name two types of threats to validity and explain them briefly.
    \item Explain the two main principles of Grounded Theory.
    \item Explain one method that supports Grounded Theory and explain why it supports it.
    \item Match the given Research Methods to Qualitative/Quantitative: 
      \begin{itemize}
        \item Content Analysis
        \item Random Sample
        \item Statistical Test
        \item Observation
        \item Survey
      \end{itemize}
    \item Match the given metrics to their respective Research Method:
      \begin{itemize}
        \item Code Churn
        \item Cyclomatic Complexity
        \item Code Coverage
        \item Code Smells
        \item Code Duplication
      \end{itemize}
    \item Study Design: You label 5000 GH Issues with Tags yourself for an AI model. This takes you 6+ Hours each day without breaks, every day, for about a month straight. Find two threats to validity, explain them, and name a possible mitigation technique for each.
  \end{enumerate}

  \begin{solution}
  
  \end{solution}
\end{exercise}

\begin{exercise}{Patterns}
  \begin{enumerate}
    \item Name two architectural patterns and describe one of them briefly.
    \item Sketch out the test pyramid and give one advantage for each type of test (no nullifying or repeating previous arguments).
    \item What is a Model Card? Give two examples of model card categories and describe them briefly.
    \item Fill in the test driven development steps. Describe two necessary steps of what to do/ensure in the fourth step.
      \begin{itemize}
        \item Step 1: Write a failing test
        \item Step 2: \
        \item Step 3: Make the test pass
        \item Step 4: \
      \end{itemize}
    \item Name two deployment patterns and explain one briefly.
    \item Match the definitions of Phenomenon, Concept, Model, Instance.
      \begin{itemize}
        \item Phenomenon
        \item Concept
        \item Model
        \item Instance
      \end{itemize}
      \begin{itemize}
        \item A representation of a phenomenon.
        \item A specific example of a concept.
        \item A generalization of a phenomenon.
        \item An observable event or occurrence.
      \end{itemize}
  \end{enumerate}

  \begin{solution}
    % TODO
  \end{solution}
\end{exercise}

\begin{exercise}{Requirements}
  \begin{enumerate}
    \item Fill in the second and third step of the requirements engineering process and explain any two steps:
      \begin{itemize}
        \item Elicitation
        \item …
        \item …
        \item Validation
        \item Management
      \end{itemize}
    \item Which of these is probability sampling not appropriate for? Developing a hypothesis, Representativeness, Generalization.
    \item Name six categories of non-functional requirements and for three give concrete examples.
    \item Name two differences between interviews and surveys.
    \item Which expression is correct?
      \begin{itemize}
        \item In an experiment, we manipulate the independent variable.
        \item In an experiment, we manipulate the dependent variable.
        \item In an experiment, we manipulate both independent and dependent variables.
        \item In an experiment, we manipulate no variables.
      \end{itemize}
    \item Name two descriptive statistics for labeling a GitHub issue dataset.
    \item Draw a UML Use Case Diagram for a bank. Customers can make transfers and optionally save them as templates. They can issue standing orders. Transfers and standing orders require a TAN to be entered. The system allows customers to log in. Bank employees can freeze accounts if they act suspiciously. Only bank managers can unfreeze accounts.
    \item Use Abbott's technique to turn the scenario from before into a static UML class diagram, including attributes and multiplicities, but no methods. Also, no aggregations or compositions are required.
    \end{enumerate}

  \begin{solution}
    % TODO
  \end{solution}
\end{exercise}



\sheet[2022]{First Exam}
\begin{exercise}{Requirements}
  \begin{enumerate}
    \item What are the differences between evolutionary and throwaway prototypes? Give two advantages for both.
    \item Explain how a binary search tree works in the context of requirements prioritization.
    \item Explain two types of requirements traceability.
    \item Explain two of the EU legal points.
    \item Draw a UML Use Case Diagram for a banking application. Customers can log in. They can transfer money but also save the template for a transfer. Both actions require the customer to do TAN. Workers can freeze accounts. Only managers can unfreeze accounts.
    \item Name four activities during requirements engineering:
      \begin{itemize}
        \item Requirements elicitation
        \item …
        \item …
        \item …
        \item …
      \end{itemize}
    \item Name two artifacts for traceability.
      \begin{itemize}
        \item SRS
        \item …
        \item …
      \end{itemize}
  \end{enumerate}

  \begin{solution}
    % TODO
  \end{solution}
\end{exercise}

\begin{exercise}{Patterns}
  \begin{enumerate}
    \item Explain the difference between concept, modeling, abstraction, and phenomena and map them to their definitions.
    \item Which of the following is the definition of an algorithm and which is the definition of a pattern?
      \begin{itemize}
        \item A solution to a problem.
        \item A solution to a class of problems
        \item A solution to a problem in a specific context.
        \item A solution to a class of problems in a specific context.
      \end{itemize}
    \item What is the pattern for using different sort algorithms dynamically?
    \item Name three types of organizations and describe one of them.
    \item Name the three different parts of the work allocation pattern. Explain one way to organize the organization and give one benefit and one disadvantage of this way.
    \item Imagine a spaceshuttle model with a speedometer, compass, thermometer, and information board. The information board has associations to the speedometer, compass, and thermometer. The spaceshuttle has an association to the information board. Apply a pattern for transferring data between the information board and the spaceshuttle and draw it into the started class diagram. Name the pattern you applied.
    \item You have an adblocker application. The interface has multiple algorithms to decide whether an element belongs to an ad or not. If the algorithm decides that it is an ad, it replaces it with a placeholder during rendering. Apply multiple patterns to solve this. Draw a class diagram. Name and justify the patterns.
    \item Name two usability patterns for mobile and two anti-patterns.
  \end{enumerate}

  \begin{solution}
    % TODO
  \end{solution}
\end{exercise}

\begin{exercise}{Research}
  \begin{enumerate}
    \item How can mixed models be defined?
    \item Describe how Grounded Theory works in two sentences. Give an example where it was used.
    \item If you were one of the Gang of Four and would like to study design patterns empirically, which method would you use to first explore the patterns and then quantify their usage in the projects? Justify your decision with one sentence each.
    \item Two groups are instructed to solve a difficult coding problem. The first group does this as pair programming, the second group alone. The task is observed by one person with a pencil and pen. Name two threats to internal validity and explain why. For each threat, describe an encounter tactic.
    \item Explain inter-rater reliability and why it is important.
    \item Are the following methods related to qualitative or quantitative research?
      \begin{itemize}
        \item Content Analysis
        \item Random Sample
        \item Statistical Test
        \item Observation
        \item Survey
      \end{itemize}
    \item Non-probability sampling is not suitable for
      \begin{itemize}
        \item Case studies
        \item Representativeness
        \item Development of hypothesis
      \end{itemize}
    \item What is the difference between structured, unstructured, and semi-structured interviews?
  \end{enumerate}

  \begin{solution}
   % TODO
  \end{solution}
\end{exercise}

\end{document}