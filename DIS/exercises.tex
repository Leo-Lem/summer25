\documentclass{article}

\usepackage[]{xrcise}

\subject{Databases and Information Systems}
\semester{Summer 2024}
\author{Leopold Lemmermann}

\begin{document}\createtitle

\sheet[2022]{Second Exam}
\begin{exercise}{Histories and Schedules}
  Given the following schedule:
  \begin{enumerate}
    \item $T_1: r(x), w(x), r(y), w(y)$
    \item $T_2: r(x), w(x), r(y), w(y)$
    \item $T_3: r(x), w(x), r(y), w(y)$
  \end{enumerate}

  \begin{enumerate}
    \item Determine if the schedule is in CSR.
    \item If not, draw the conflict graph.
    \item Rank the following: Full > FSR > VSR > CSR > OCSR > COSCSR > Serial.
  \end{enumerate}

  \begin{solution}
    % TODO
  \end{solution}
\end{exercise}


% Aufgabe 2: Concurrency Control
% • Es war ein Konflikt Graph gegeben mit 6 Transaktion die wild Abhänigkeiten und Zyklen
% haben. Es mussten zwei Strategien angegeben werden wie Deadlocks aufgelöst werden
% können und die TA(s) angeben die zu einem Zyklenfreien Graphen führen.
% • Erkennen von den Kanonischen Problemen (Dirty Read, Lost Update, Non Repeatable
% Read, Phantom Problem) in Schedules.
% • Eine Aufgabe mit der Abfrage ob ein Shedule in 2PL, S2PL und SS2PL. Der gegebene
% Schedule musste auch in 2PL mit read und write Locks aufgeschrieben werden.
\begin{exercise}{Concurrency Control}
  Given the following conflict graph:
    % TODO: Insert graph here (6 transactions with wild dependencies and cycles)

  \begin{enumerate}
    \item Give two strategies to resolve deadlocks and name the transactions that lead to a cycle-free graph.
  \end{enumerate}

  \begin{solution}
    % TODO
  \end{solution}
\end{exercise}


% Aufgabe 3: Logging und Recovery
% • Es wurde ein Log-File gegeben
% • Transaction Consistent Checkpoint (TCC), Non-Atomic/Steal/NoForce
% • Es mussten Winner/Loser TA gefunden werden
% • Angeben wann die REDO/UNDO Recovery startet.
% • Angeben wie oft es abgestürtzt ist (anhand des Log?). Hier sind wir uns etwas unsicher wie
% das zu verstehen ist, grundsätzlich wurde in der Aufgabenstellung angegeben, dass der
% Absturz stattgefunden hat aber es gab eine Compensation Log Records in der Log-Datei.
\begin{exercise}{Logging und Recovery}
  \begin{enumerate}
    \item 
  \end{enumerate}

  \begin{solution}
    % TODO
  \end{solution}
\end{exercise}


% Aufgabe 4: NoSQL
% • Angabe der 4 Kategorisierung von NoSQL-Datenbanken mit je einem Beispiel
% • Allgemeine Fragen zu NoSQL und den eigenschaften (ja/nein)
% • Abfrage von CAP und was es bedeutet und was das Problem damit ist.
% • Two Phase Commit Protocol und wie es funktioniert
% ◦ Der Coordinator stirbt bevor er den COMMIT-Befehl gesendet hat, er hat aber schon ins
% Log geschrieben, dass er senden möchte. Was passiert?
% ◦ Der Coordinator kommt wieder zurück, was kann er unternehmen (es wurde schon
% geschreiben, dass der Commit senden möchte)
% ◦ Ein Agent findet heraus, dass ein anderer Knoten den COMMIT-Befehl bekommen hat.
% ◦ Der Coordinator kommt wieder, hat aber kein Log für das commit geschrieben, obwohl
% er alle READY-Nachrichten erhhalten hat. Was passiert?
\begin{exercise}{NoSQL}
  \begin{enumerate}
    \item 
  \end{enumerate}

  \begin{solution}
    % TODO
  \end{solution}
\end{exercise}


% Aufgabe 5: Data Warehouse
% • Es war ein Schema gegeben, es musste bestimmt werden welcher Typ das Schema ist
% (Galaxy).
% • Es musste dem Schema eine Tabelle hinzugefügt werden die aggregierte Daten enthält.
% • Es musste eine SQL Abfrage geschrieben werden die für die Jahre 2000 bis 2010 den
% Durchschnitt für jedes Jahr von einem Fact bestimmt.
% • ETL musste beschrieben werden, eventuell musste auch 3 Typen von Transformationen
% beschrieben werden die im ETL-Prozess passieren.
% • Wurde CUBE(A), GROUPING SET(C), ROLL UP (B) gegeben und man musste angeben
% welche grouping Sets daraus entstehen.
% • Welche funktionale Abhängigkeiten muss gegeben sein, dass alle angegebenen grouping
% Sets auch entstehen.
\begin{exercise}{Data Warehouse}
  \begin{enumerate}
    \item 
  \end{enumerate}

  \begin{solution}
    % TODO
  \end{solution}
\end{exercise}


% Aufgabe 6: Data Mining
% • K-NN mit Manhatten distanz, K = 5. Es war ein Kartesisches Koordinatensystem gegeben in
% dem verschiedene Punkte angegeben waren und entsprechenden Klassen zugeordnet. In der
% Aufgabenstellung wurden weitere Punkte gegeben die Klassifiziert werden sollten.
% Es war eine Tabelle gegeben in die die nahen 5 Punkte eingetragen werden konnten mit
% Distanz. Daraus sollte ermittelt werden welche Klasse die neuen Punkte sind.
% • Es wurde gefragt wie die Abstände von Klassifizierungen bestimmt werden können (Single
% Linkage, Complete Linkage, Group Average, Canonical Entity)
% • Malen eines Dendrogramms
\begin{exercise}{Data Mining}
  \begin{enumerate}
    \item 
  \end{enumerate}

  \begin{solution}
    % TODO
  \end{solution}
\end{exercise}


% Aufgabe 7: Big Data Analysis
% • Es wurden einfache Wissensfragen gestellt, hier reicht vollkommen die Vorlesung einmal
% gesehen zu haben.
% • Es wurden auch zu zwei echten Datenbanken, die erwähnt wurden, Fragen gestellt (Flink
% und Spark).
\begin{exercise}{Big Data}
  \begin{enumerate}
    \item 
  \end{enumerate}

  \begin{solution}
    % TODO
  \end{solution}
\end{exercise}


\end{document}
